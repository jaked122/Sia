\newcommand{\numhosts}{192}
\newcommand{\storageperhost}{256GB}

\newcommand{\childrenperparent}{192}

\newcommand{\fortynine}{$\frac{n}{2} - 1$}
\newcommand{\fiftyone}{$\frac{n}{2} + 1$}

\newcommand{\filechecksize}{256 bytes}

\newcommand{\maxcorruption}{80\%}
\newcommand{\inversemaxcorruption}{20\%}
\newcommand{\maxredundancy}{5}
\newcommand{\minuncorruptedhosts}{$\frac{n}{5}$}

\newcommand{\maxslicesize}{1mb}

\documentclass[twocolumn]{article}

\begin{document}

\title{Sia : Decentralized, Compensated, Self-Repairing Computer Storage}

\author{
{\rm David Vorick}\\
Rensselaer Polytechnic Institute
\and
{\rm Colin Rice}\\
Rensselaer Polytechnic Institute
}

\maketitle

\subsection*{Abstract}
Sia is a decentralized platform for computer storage.
Host computers contribute storage and bandwidth in return for Sia's internal cryptocurrency - Siacoin.
In addition to receiving compensation from the users, host computers also mine siacoins in proportion to the volume of storage they contribute.
Clients acquire siacoins (either by being hosts or through a currency exchange) to exchange for storage space and download bandwidth on the Sia network.
The cost of renting storage is set using a supply and demand algorithm.
Sia is a cryptosystem that is secure even when \fortynine \space of the hosts are dishonest and malicious.
Sia is built out of many small blockchains arranged in a tree structure.
Blockchains operate using a concensus algorithm instead of proof-of-work.
To join a blockchain, you must have proof-of-storage equivalent to \storageperhost.

\section{Introduction - Nontechnical}

Sia is a decentralized platform for storing data on the internet.
With only a few kilobytes of overhead, any file can be stored on the Sia network.
Sia supports anonymity, and guaruantees uptime for the files.
Files can be fetched with only a few hundred milliseconds of latency.
Files are random-access, meaning a piece of the file can be downloaded without fetching the whole file.
All files are seeded by \numhosts \space peers, meaning files can be uploaded and downloaded at high speed.
Sia minimizes file redundancy while maximizing speed and security, making Sia competitive with centralized services.

Files are stored by hosts in exchange for Siacoins.
The price of storage is set algorithmically according to supply and demand.
Siacoins are mined on the Sia network and are the only currency that can be used to purchase Sia storage.
This constraint provides stability to the price of the siacoin - the price of the coin is tethered to the price of Sia storage.
Siacoin is a superset of Bitcoin.

Files can be stored in two modes.
\begin{itemize}
	\item Immutable - once uploaded, the file can never be altered, and the size of the file is constant.
	At the end of the file, there is a pointer to another file, which can contain information about updates to the original file.
	This pointer can be null.
	Immutable files are appropriate for shared data.
	\item Mutable - the contents of the file can be changed by anyone with permission.
	The original contents of the file are permanently lost after a change is made.
	Mutable files are appropriate for personal/private data.
\end{itemize}

Files have a balance of siacoins, which is used to pay the hosts that store the file.
Each block, a volume of siacoins is removed from the balance according to the current price of storage.
When the balance runs out, the file is deleted.
Anybody can increase the balance at any time by sending more siacoins to the file.
There is no way to remove a file from the network aside from waiting until the balance depletes.
There is no way to extract coins from the balance - once sent to a file, siacoins are locked into a single purpose.
These constraints protect against censorship.

Sia can be used as a more efficient replacement for many of todays services.
These services include Bittorrent, Bitcoin, and Filelockers.
Sia can also be used as a place to host embedded web content.
On fast connections, Sia can be used as a replacement for hard drives - computers would only need enough local storage to boot and connect to the internet.

One major advantage of Sia is that all shared files only need to be hosted once, yet no central service needs to be trusted or depended on.
Another advantage of Sia is that you only need to rent as much storage as you are actively using - more storage can be rented on-the-fly.

% I may have included too much information in this section, but it's important to give an overview of the whole network. Pretty much every topic discussed needs to be covered, but it may not need to be so detailed.
\section{Introduction - Technical}

Sia is built out of many small blockchains of \numhosts \space hosts each.
These blockchains participate in a deterministic consensus algorithm instead of using a block mining algorithm such as proof-of-work.
Blockchains are assembled randomly from hosts on the network - hosts cannot control which blockchain they are participating in.
Each host must be contributing \storageperhost \space to the network in order to be placed in a blockchain.
A single machine with lots of storage can operate as many hosts simultaneously - as long as each host has \storageperhost \space of unique storage.
Blockchains will be honest as long as at least \fiftyone \space of the hosts within the blockchain are honest.

Storage is only valid once there are files uploaded to the network to store.
This is because the proof-of-storage algorithm requires actual data on which to perform verifications.
The proof-of-storage algorithm makes use of the fact that data is redundant - each host is storing different Reed-Solomon coded pieces of the files uploaded to the network.
By gathering file pieces from each host, the network can use the pieces to check for corruption.

Each blockchain has a set of files which it tracks.
As files are added and deleted, the network performs load-balancing.
When the total amount of available storage on the network drops below a certain threshold, a new blockchain is created.
The load balancing algorithms then pick a bunch of files to store on the new blockchain.
If the network rises above a certain threshold of available storage, a blockchain is chosen to be deleted.
That blockchain then load-balances every file to other blockchains and terminates.

If there are many hosts waiting to join blockchains, and not enough new files for the hosts to store, the price of storage will drop.
Dropping the price not only affects supply and demand, it also means that existing files will be on the network for longer.
If the amount of available storage on Sia is low and there are not many hosts for creating new blockchains, the price of storage will increase.
Increasing the storage not only affects supply and demand, it also means that existing files will be deleted faster.

% I need help writing this paragraph
Blockchains are organized into a tree that manages the needs of the network.
Parent blockchains have aggregate information about their children - such as how much data is being stored by each child.
Only the leaf blockchains actually store files - all the other blockchains are designed to be very lightweight.
The root blockchain determines the network price, and also knows how many new hosts there are and how much free space there is on the network.
When the root blockchain needs to do something such as create new blockchains, it delegates the work to its children.
Those children delegate work to their children, until the action hits the leaf layer, where it is carried out.

% This one too
The same happens in the reverse direction.
When a leaf blockchain deletes a file, it tells its parent, who then updates its aggreagate resource.
Parent blockchains are in charge of load-balancing their children, and they wait for their parents to load balane them.

% And this one too - heck, probably all of them
Existing systems use only a single blockchain, where very node is aware of every event on the network.
This will not scale well, especially in a system attempting to track potentially billions of files and hundreds of millions of daily downloads.
Using a tree means that a leaf blockchain only needs to be aware of its parent, and its grandparent, all the way up to the root.
Because parent blockchains are very lightweight, and because there are \childrenperparent \space children per layer, this is very little information to track.
Because parent blockchains track aggregate resources, a leaf blockchain can be certain the the network approves of its funds without needing to know the state of the entire network.

Siacoins and wallets are treated in much the same way as files.
Wallets consume space on the network, and get charged for the space they consume.
When wallets send money to a different blockchain, the transaction propagates through the tree.
Information is aggregated, meaning that the parent blockchain will only say how many coins have moved between children - it will not say which wallet the coins are from or which wallet the coins are for.
This compression is necessary to keep the amount of information sent through the root blockchain at a minimum.
This compression also means that there needs to be another way to determine which wallets incoming siacoins are for.

This is accomplished by having leaf blockchains directly talk to eachother.
When a wallet in one leaf blockchain is sending money to a wallet on another leaf blockchain, the first blockchain will send a message indicating that coins are going to be sent.
The second blockchain then claims to its parent the correct volume of blockchains, using the sent message as proof.
The parent then knows how many coins to claim.

Blockchains can look up the direct communcation address of other blockchains through a DHT.
All leaf blockchains participate in the DHT, and only leaf blockchains participate in the DHT.
This allows blockchains to find eachother in log(n) time without sending high volumes of messages through their parents, and without needing to store the direct communication address of every blockchain in the network.

Sia has an indictment framework - every action by a host or blockchain is signed and verified by a large number of other hosts or blockchains.
If any host or blockchain is caught performing an illegal action, that host or blockchain is fined and thrown from the network.
To make fines successful, fines must be equal to the amount of damage caused, and hosts/blockchains must always have enough siacoins to pay the maximum possible fine that they could incur (called a deductible).

Sia assumes that \fiftyone \space of the hosts on the network are honest.
If \fortynine \space of the hosts are corrupted, a large volume of blockchains will also be corrupted.
Parent blockchains are not immune from this corruption.
The indictment framework, plus a series of checks and balances insures that the network can operate without loss even if as many as \maxcorruption \space of the blockchains are dishonest, and even if as many as \maxcorruption \space of the hosts in a particular blockchain are dishonest.

If only \fortynine \space of the hosts are corrupted, it is extremely unlikely that the network will ever have a blockchain with more than \maxcorruption \space of the hosts corrupted, and it is also extremely unlikely that an attacker will corrupt more than \maxcorruption \space of the blockchains in the network.

\section{Consensus Blockchains}

Concensus blockchains are formed from \numhosts \space hosts all storing \storageperhost.
Each blockchain hosts a set of files and wallets that are unique to that blockchain (each object only appears on a single blockchain).
Blockchains have a 'state', which is a representation of the current status of the network.
The state is updated in blocks, which are produced in a concensus algorithm.

Blocks are composed of 'heartbeats', which is an update package from a host.
Each block, every host must submit a heartbeat.
Heartbeats contain storage proofs, keepalive information, and general network transactions and updates, and will be discussed in greater detail later.

The blockchain algorithm assumes that up to \fortynine \space of the hosts in the blockchain are dishonest and malicious.
Each block, honest hosts must be guaranteed to have their heartbeat included.
Furthermore, every honest host must be guaranteed to include the same set of heartbeats, so that the blockchain can maintain a state of consistency.
These things are guaranteed by the following algorihtm:

\begin{enumerate}
	\item All hosts send their heartbeats to all other hosts.
	\item All hosts tell the other hosts which heartbeats they received, and sign that they received these heartbeats.
	In this step, hosts will fill in gaps in each other's list of heartbeats.
	\item All hosts tell the other hosts which heartbeats they know were seen by at least \fiftyone \space of the hosts on the network, and include the list of signatures proving that the hosts actually saw the heartbeats.
\end{enumerate}

By the end of the third step, the honest hosts will be guaranteed to have all received the same set of heartbeats that have been seen by \fiftyone \space of the hosts on the network.
All hosts will therefore produce the same block.
The process can then restart for the next block.

\section{Concensus Algorithm Proof}

Assumptions:
\begin{itemize}
	\item 51\% of the network is honest.
	\item All honest hosts can communicate freely. (no DOS attacks)
	\item The network is synchronized, meaning all hosts can complete each step within a known limited timeframe, and hosts can self-correct for drift
\end{itemize}

In step 1, every honest host will get heartbeats from every other honest host.
If \fiftyone \space hosts are honest, every honest host is guaranteed to have their heartbeat seen by at least \fiftyone \space hosts.

In step 2, every host tells each other host which heartbeats they have seen, and attatch a signature that they saw the heartbeat.
This means that any heartbeat that was seen by any honest host will be seen by every other honest host.

In step 3, every host tells each other host which heartbeats they got which were seen by at least \fiftyone \space hosts.
This is to guarantee that the heartbeat was seen by at least 1 honest host in the previous step.
If it was seen by at least 1 honest host, then every honest host is guaranteed to have received it in this step.
Therefore, every honest host is guaranteed to include every heartbeat that was seen by \fiftyone \space hosts in the second step.

\section{Dishonest Blockchains}

In a network of randomly assembled blockchains where \fortynine \space of the hosts are compromised, around half of the blockchains are also likely to be compromised.
This means that honest hosts need to be able to detect when their blockchain has been corrupted and need a way to restore order, protect wallets, and perform damage control on any files.

If an attacker controls \fortynine \space of the hosts on the network, then the attacker has less than a (1 times 10 to the 16th) chance of compromising a blockchain by more than \maxcorruption.
Compromised blockchains are okay as long as \inversemaxcorruption \space honest hosts can detect that they are participating in a dishonest blockchain and work together to protect the vital information stored on the compromised blockchain.

\section{File Storage Layout}

Each blockchain has \numhosts \space and \storageperhost \space per host.
Each \storageperhost \space block of data is split up into many smaller pieces, which are to be called slices.
Slices have no minimum file size, but have a maximum file size of \maxslicesize.
The collection of slices on a single block of data is called a stack.

The slices in each stack correcspond to each other.
For example, if the first slice on one host is 45kb, then the first slice on every single host is also 45kb.
These inter-host sets of congruent slices are called rings.

Files are stored on multiple hosts using Reed-Solomon coding, which is maximum distance separable.
First the user picks redundancy settings, which is what volume of the hosts need to retain their piece of the file in order for the file to be recovered.
This is m of n recovery, where n equals \numhosts.
If m is 1, then the file will be duplicated on every single host, and can be recovered from only a single host.
The redundancy is then \numhosts \space, meaning that a 1GB file will take \numhosts GB of storage on the Sia network.
If m is \fiftyone, then the file can be recovered from any \fiftyone \space hosts on the network, and the redundancy will be slightly less than 2.
A 1GB file will therefore consume slightly less than 2GB of Sia storage.

Files have header data that includes the hash for each slice in the ring and the value of m.
The value of m can be stored as a single byte.
Each hash will be 32 bytes and there are \numhosts pieces.
The network assumes that under any circumstances, a blockchain will have at least \inversemaxcorruption hosts, so the redundancy on the header data will be kept at \maxredundancy.
Altogether, this means that there is 30kb of overhead for a file.

Part of the heartbeat algorithm performs keepalives on hosts and verifies that they are storing their pieces as promised.
If a host goes offline or corrupts their data, the network can use the Reed-Solomon coding to repair the file within minutes - no interaction from the user needed.

It is left to the user to pick optimal settings.
When m is small, there is a higher chance that a malicious attacker can control enough of the network to knock the file offline.
If a file is high risk or needs to be guaranteed safe against powerful attackers, setting m to \minuncorruptedhosts \space will be essentially guaranteed to protect the file.

\section{Proof of Access}

Each heartbeat, a host has to provide a proof that they have access to the file pieces that they are storing.
Each block, a random \filechecksize \space segment of the storage stack is chosen to be included in the heartbeat.
This segment will correspond to a portion of a file in a file ring.
Each host will include their portion of the file in their heartbeat, which means that in the block there is enough information to fully reconstruct a small portion of the file.
This will also reveal which hosts have the wrong data.

Only \filechecksize \space is requested each block because requesting more would greatly increase bandwidth requirements per block.
This small amount is sufficient because it is randomly selected.
If a host is missing a substantial portion of the file, it will only be a matter of time before some part of the missing portion is randomly selected.
This does mean that tiny corrupted sections of file can go undetected for a long time, but even small corruptions will eventually be detected and are unlikely to prevent file recovery.

We can assume that a malicious host will receive all of the honest heartbeats before submitting their own.
This means that, given sufficient redundnacy, a malicious host will be able to rebuild their piece of the file every step from the other heartbeats, can cna perform proof of access without actually storing anything.
To prevent that, this process is broken up across two stages.

In the first stage, hosts only submit the hash of the chosen segment prepended with their id.
This forces hosts to lock in what their file piece is before they see any information revealed from the other hosts.
In the second stage, hosts reveal what string produced the hash.
If they reveal a string that does not hash to the string submitted in stage 1, the host is considered corrupt.

\section{Proof of Storage}

Storage is considered valid on the network if it follows two constraints:

\begin{itemize}
	\item It costs money to host files on the network.
	\item The storage is unique to the network.
\end{itemize}

The assumption for the first bullet point is that if a file costs money to upload, nobody will upload files that are not useful to them in some way.
A host will not upload a fake file to himself or the network if the act of doing so forces the host to operate at a net loss.
The second bullet point means that the host is not using Sia as it's source of storage, or that the host is not using a source that uses Sia as a source for storage.

The first point is protected by randomly collecting together swarms.
Any swarm will be composed of random sets of nodes, and will not have more than \maxcorruption \space dishonest hosts.
This means that any host attempting to use themselves as exclusive storage will have to host their files on at least \inversemaxcorruption \space honest hosts.
As long as the profit from mining on \maxcorruption \space of a fake file does not exceed the loss of paying for hosting on the \inversemaxcorruption honest hosts, there will be no incentive to do fake mining in this manner.
Therefore, a hard cap will be placed on the amount of coins that can be mined to satisfy this condition.

The second point is protected by the cost of downloads on the network.
Downloading a file costs money, and if you are to complete the proof of access steps in the heartbeat, you need to download a fraction of the file each heartbeat.
As long as the download system is constructed such that the cost of downloading repeatedly to perform heartbeat access proofs is greater than the reward from mining, this attack will not be a problem.
The system for pricing downloads has currently not been created, but it will be created with this constraint in mind.

% game theory stuff about why it's not beneficial for an incomplete minority to collaborate until they hit a critical mass

Protection against corrupted uploads: it is possible that a host has the file that they were uploaded and yet still fail the corruption test.
They will be indicted for storing incorrect data, and the indictment will be incorrect.
Hosts can prove themselves innocent by uploading the entire file piece as well as the hash of the file piece that was created when the file was uploaded.
If the hosts file piece matches the hash, and the hosts indicted slice correctly matches the file piece, then the file is determined to have been corrupted at upload.
The host is acquitted and the file piece is corrected, the host then has the correct file piece.

\section{The Problem of Delegation}

\section{Entropy}

Each block, a random file segment must be selected.
This means that the blockchain needs an unpredictable way to produce entropy, otherwise a dishonest host could look ahead and only store the pieces that it knows will be "randomly" selected.
Each block, every host will be required to produce a random string that is 32 bytes in length.
There are no requirements on how this is to be done, but it is assumed that honest hosts will have a legitamate method for producing random strings.

Just like the proof of access, the random strings will have two stages.
In the first stage, only the hash of the randomly generated string will be revealed.
In the second stage, the actual random string will be revealed, and then all of the random strings will be appended deterministically and then hashed to produce the final random string for the block.

By determining entropy this way, we guarantee that if even a single host produces a random string, the final string will also be random.

This method of generating entropy is vulnmerable to an attack: dishonest hosts can intentionally withdraw from the network to influence the outcome of the random number.
The way to prevent this is to make sure that the penalty for missing a heartbeat is more expenive than favorably generating entropy for a potential 80\% of dishonest hosts.
There will be a penalty for missing heartbeats, and it will sufficiently satisfy this attack prevention mechanism.

\section{Full Heartbeat}

\section{A Tree of Blockchains}

Explain that blockchains are organized into trees, operate on assumption of 51\% dishonest blockchains, talk about parent and child relationships. Talk about detecting a comporomised parent, and immediately replacing them to minimize damage to the network.

\section{Dishonest Blockchains}

Talking about how hosts need to be forcefully removed when hosts are dropped, talking about honest hosts resigning, etc.

\section{Aggregate Transactions}

Explain how aggregate transactions work, aggregate storage. Attack vectors such as dishonest swarms sending dishonest spends. Prove detection of dishonesty and integrety of aggregation given 51\% honest swarms.

\section{Aggregation DHT}

Explain the DHT that allows blockchains to talk to eachother without flooding the parent with communication.

\section{Stuff to be integrated later}

There is an entire scripting system, as well as potentially a cacheing system or something like that. I would really like to see an emergency home for dynamic pages built into this system (even if it's a crappy and slow home) because Tor has been compromised and currently there is nowhere else to go.

\section{Implications}

An entire section dedicated to talking about the implications of universally available cheap cloud storage, especially when it's encrypted and decentralized.
Then I can also talk about the implications of the scripting system.
We can expand here on services that might succeed on Sia, being able to take advantage of its unique features.
Bittorrent, onion routing, anonymous money, etc.

An entire paragraph needs to be dedicated to talking about the economic implications of the cryptocurrency.
Min and max price, etc.

\end{document}
